\chapter{Aim of the Experiment, Theoretical Basis }

The aim of this experiment is to determine the lifetime of positronium, an unstable atom made of a positron and an electron in plexiglass. The positron's lifetime is characterized by it's decay possibilities, which are further discussed in the following text. The experimental setup also allows to conveniently determine the speed of light, due to the great time resolution of the detector setup.

\section{$\beta$-decay}
$\beta$-decay is a radioactive decay in which the mass number of the nucleus stays constant, but the ordinal number changes by one unit. The $\beta$-decay is relevant for this experiment because the $\beta^+$-decay of $^{22}$Na is used to determine the lifetime of the positron/positronium. There are a few different cases of $\beta$-decay, which are portrayed in the following.
\subsection{$\beta^+$-decay}
The $\beta^+$-decay is a decay from a proton into a neutron, a positron, and an electron-neutrino
$${}_{{1}}^{{1}}{\mathrm {p}}\to {}_{{0}}^{{1}}{\mathrm {n}}+{\mathrm {e}}^{{+}}+\nu _{e}\mathrm{.}$$

\subsection{$\beta^-$-decay}
The $\beta^-$-decay is a decay from a neutron into a proton, an electron, and an electron-antineutrino
$${}_{{0}}^{{1}}{\mathrm {n}}\to {}_{{1}}^{{1}}{\mathrm {p}}+{\mathrm {e}}^{{-}}+\overline {\nu }_{e}\mathrm{.}$$
This decay is used in the KATRIN experiment to determine an upper limit of the neutrino mass. 
\subsection{Electron capture}
The electron capture is counted as a $\beta$-decay, although no $\beta$-particle is emitted. In this case, a proton of the nucleus is changed into a neutron. With this change, an electron close to the nucleus is destroyed and a neutrino created and emitted
$${\displaystyle {}_{Z}^{A}\mathrm {X} +\mathrm {e} ^{-}\to {}_{Z-1}^{A}\mathrm {Y} \mathrm {+} \nu _{e}}\mathrm{.}$$

\subsection{decay of a free neutron}
A free neutron can decay
$${\hbox{n}}\to {\hbox{p}}+{\hbox{e}}^{-}+\overline {\nu }_{{{\mathrm {e}}}}\mathrm{.}$$
However, the average lifetime of a free neutron is significantly greater than the average time it takes for a nucleus to capture the free neutron, so this decay is seldom relevant.
%\subsection{Inverted $\beta$-decay}
%With the inverted $\beta$-decay a proton decays into a neutron and a positron, by reacting with a neutrino
%$${\displaystyle \mathrm {p} +{\overline {\nu }}_{e}\to \mathrm {n} +\mathrm {e} ^{+}}\mathrm{.}$$
%Using this decay, the first experimental verification of neutrinos succeeded in 1956.
\section{Annihilation}
Annihilation is a process in particle physics in which a particle collides with its antiparticle to produce other particles due to conservation of energy and momentum. In the case of an electron and a positron they both annihilate in at least one gamma quantum. Creating at least two gamma quanta or more is always allowed whereas pair annihilation for just one is not. The creation of one single photon is only possible in matter in which conservation of energy an momentum can be obtained by momentum transfer to a collision partner. In two-quanta decay the electron-positron pair decays into two photons with opposite momenta. To conserve the energy, each of these photons have the energy of $511\,$keV, which corresponds to the electron rest mass. Another form of decay is the three-quanta decay, in which three photons are created that have arbitrary energies and the emission angles are in one plane. Due to the low cross section of the one-quantum decay it is less likely to take place in comparison to the two-quanta decay and therefore will be neglected in the following. \\
Besides conservation of energy and momentum there is the spin of the two particles that have to be considered and has been neglected in the previous discussion. Electrons and positrons are leptons, which have the spin $S=\frac{1}{2}$. The two possible states, in which they can exist have a total spin of $S=0$ and $S=1$ and are called singlet and triplet state. Photons have the spin $S=1$ and are emitted parallel or antiparallel to the direction of flight. Two photons emitted in opposite directions leads to a total spin of 0 or 2. Due to conservation of total angular momentum, the singlet state decays into two quanta whereas the triplet state decays into at least three photons which means that it is a process of higher order and therefore less probable. The ratio of the two decay possibilities
\begin{equation}
    \frac{\sigma_{3\gamma}}{\sigma_{2\gamma}}=\frac{1}{372}
    \label{ration decay}
\end{equation} shows that it is less probable for the triplet state to take place, which is why it will be neglected in the following.



\section{Positronium}
Another possible result of electron-positron interaction besides annihilation is the formation of an exotic atom called positronium. It is similar to a hydrogen atom with the difference that the positron replaces the proton. The electron and the positron orbit around their common center of mass. The calculation of the energy states is analogous to the hydrogen atom except now the reduced mass of the atom is given by the reduced mass of the positronium. The latter is equal to half the mass of an electron therefore the binding energy is half that of hydrogen $E=6.8\,$eV. There are two spin states, the singlet state which is called para-positronium and the triplet state is called ortho-positronium. A comparison of the two lifetimes of the states shows that the average lifetime of the ortho-positronium is about a thousand times the one of para-positronium. Therefore, ortho-positronium rarely decays from excited state because lives long enough for the excited state to transfer to the ground state.
Besides the occurence of positronium in gases, it can be created in matter.. Before the formation of a positronium, the postiron is very fast, consequently having a high kinetic energy. Due to inelastic collisions with electrons, they decelerate, but still are still able to ionize the atoms. Within $10^{-12}\,$s the velocity of the postirons has reduced to the point where their kinetic energy is in the eV range. Now annihiltion and formation of positronium is possible. There is a certain energy range $\Delta E$ where no inelastic collisions are possible, only the formation of positronium. The minimum limit is given by $E_\text{kin}=V-6.8\,$eV with $V$ being the ionization energy of the atom and $6.8\,$eV representing the binding enegry of positronium. The maximum energy limit is the lowest excitaion energy $E_\text{a}$ of the atom. Thus, the energy range is $$\Delta E=E_\text{max}-E\text{min}=E_\text{a}-(V-6.8\,\text{eV})$$ also called the Ore-Gap.\\
To detect positronium it is not sufficient to look at the two- or three-quantum decay, because these quanta can come from, free decay. It is useful to determine the ratio for free decay which is given by equation \ref{ration decay} and the ratio for the creation of positronium which is 
\begin{equation}
    \frac{\sigma_{ortho}}{\sigma_{para}}=\frac{3/4}{1/4}=3 ~.
    \label{ratio formation}
\end{equation}
This indicates that during the formation of positronium the chance for orhto-positronium to be formed is three times higher than for para-positronium because the triplet state has three spin states. However, in reality the ratio is lower than given in \ref{ratio formation}. The reason for this is that, due to the long lifetime, the ortho-positronium interacts with its environment and can transform into para-positronium.\\
When looking at postronium in solids one can see that there are processes, so-called pick-off processes and conversion, that effect the lifetime of positronium. The first process describes the interaction of positronium with electrons from molecules of the solid or with the inner-molecular magnetic fields. Now the ortho-positronium has a collision partner that absorbs momentum and angular momentum leaving the possibility for the triplet state to decay into two quanta. The other process is the conversion between para- and ortho-positronium, which is possible by exchanging electrons from the positronium with electrons of the matter. The conversion in the other direction is just as likely to happen. However, the conversion probability is higher than the decay probability of the triplet state and lower than that of the singlet state. Therefore, the rate of creation of ortho-positronium is less than of annihilation. 