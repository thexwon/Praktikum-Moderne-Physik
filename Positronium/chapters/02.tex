\chapter{Measurement Principle and Experimental Setup}
\section{Measurement Principle}
As there is a low chance of a three-quantum decay occurring inside a molecular substance, we can conclude that the observed positron annihilation  emits radiation quanta of either a fixed \SI{0.511}{MeV} from the singlet state positrons, or a continuous spectrum with the same maximum energy from the pick-off processes.
In order to determine the lifetime of the positron, observing a $\beta^+$-decay of which one of the products emits a $\gamma$ quantum is necessary. The detection of such a gamma quantum marks the start of the positrons lifetime, and also the time of possible formation for positronium, after a deceleration time of roughly   
\SI{e-12}{s}. In this experiment, $^{22}$Na is used
as the source of $\beta^+$-particles. During the decay, the  daughter nucleus is $^22$Ne preferentially left in an excited state. The transition to the ground state has an average lifetime of \SI{e-13}{s} and releases a $\gamma$-quantum with the energy of \SI{1.276}{MeV}. As the average lifetime of the transition is below the measurement accuracy, it is taken as instantaneous. 

\section{Experimental Setup}

\begin{figure}[H]
    \centering
    \includegraphics[width=110mm,scale=0.5]{Positronium/include/positronium.png}
    \caption{Sketch of the experimental setup} 
    \label{fig:Versuchsaufbau}
\end{figure}

The task of this experiment is to measure the average lifetime of positronium in plexiglas. $^{22}$Na is used as the source for $\beta ^{+}$-particles. It decays into $^{22}$Ne emitting a $\gamma$-quantum, which is detected and marks the start signal for the TPC (time to pulseheight converter). Another detector registers the entire energy spectrum. The $\gamma$-quanta are detected by plastic scintillators arranged in a way that the start and stop quanta fly apart at an angle of $180$°. It should be mentioned that the start quanta should arrive $1\,$ns before the stop quanta. Therefore, the stop quanta will have a delay of about $2$\,ns in every measurement.