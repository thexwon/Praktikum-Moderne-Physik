\chapter{Versuchsvorbereitung}


\section{Ziel des Versuchs, Theoretische Grundlagen}
Die Parität galt lange als Erhaltungsgröße, bis 1956 Chien-Shiung Wu mit dem nach ihr benannten Wu-Experiment der Nachweis für die Paritätsverletzung gelang.\\
In diesem Versuch wird die Paritätsverletzung nachgewiesen, indem das nicht-verschwinden eines Pseudoskalars, der Helizität, nachgewiesen wird. Dafür wird über die Polarisation von Bremsquanten die Helizität von Elektronen gemessen, die bei einem $\beta$-Zerfall emittiert werden. 
\subsection{$\beta$-Zerfall}
$\beta$-Zerfälle sind Kernzerfälle, bei denen die Massenzahl des Kerns erhalten bleibt, die Ordnungszahl aber um eins erhöht oder erniedrigt wird. Der $\beta$-Zerfall ist für dieses Experiment relevant, da er Prozess der schwachen Wechselwirkung ist und sich somit zur Untersuchung der Paritätsverletzung eignet. Die verschiedenen Fälle dieser Zerfallsart werden im folgenden erläutert.
\subsubsection{$\beta^+$-Zerfall}
Beim $\beta^+$-Zerfall zerfällt ein Proton in ein Neutron, ein Positron und ein Elektron-Neutrino.
$${}_{{1}}^{{1}}{\mathrm {p}}\to {}_{{0}}^{{1}}{\mathrm {n}}+{\mathrm {e}}^{{+}}+\nu _{e}$$

\subsubsection{$\beta^-$-Zerfall}
Beim $\beta^-$-Zerfall zerfällt ein Neutron in ein Proton, ein Elektron und ein Elektron-Antineutrino.
$${}_{{0}}^{{1}}{\mathrm {n}}\to {}_{{1}}^{{1}}{\mathrm {p}}+{\mathrm {e}}^{{-}}+\overline {\nu }_{e}$$
\subsubsection{Elektroneneinfang}
Der Elektroneneinfang wird zu den $\beta$-Zerfällen gezählt, obwohl dabei keine $\beta$-Strahlung emittiert wird. Hier wird ein Proton des Kerns in ein Neutron umgewandelt. Dabei wird ein Elektron einer kernnahen Schale zerstört und ein Neutrino erzeugt und emittiert. 
$${\displaystyle {}_{Z}^{A}\mathrm {X} +\mathrm {e} ^{-}\to {}_{Z-1}^{A}\mathrm {Y} \mathrm {+} \nu _{e}}$$

\subsubsection{Zerfall eines freien Neutrons}
Ein freies Neutron kann zerfallen. 
$${\hbox{n}}\to {\hbox{p}}+{\hbox{e}}^{-}+\overline {\nu }_{{{\mathrm {e}}}}$$
Die Lebensdauer eines freies Neutrons ist allerdings deutlich größer als die durchschnittliche Zeitdauer bis zur Aufnahme des Neutrons in einen Atomkern, wodurch dieser Zerfall selten eine Rolle spielt.
\subsubsection{Inverser $\beta$-Zerfall}
Beim inversen $\beta$-Zerfall wird ein Proton durch Reaktion mit einem Neutrino in ein Neutron und Positron umgewandelt.
$${\displaystyle \mathrm {p} +{\overline {\nu }}_{e}\to \mathrm {n} +\mathrm {e} ^{+}}$$
Durch diesen Zerfall gelang 1956 der erste experimentelle Neutrinonachweis.

\subsection{Parität und Pseudoskalare}

Der Paritätsoperator erwirkt eine Inversion der Koordinaten. 
$$P\Psi_{(\vec{r})} = \Psi_{(-\vec{r})}$$
Die Paritätsoperation ist einer Spiegelung und nachfolgender Drehung um $180^\degree$ äquivalent.
Da die zweifache Anwendung des Paritätsoperators auf einen Zustand wieder den Zustand selbst ergibt, sind die Eigenwerte einfach zu ermitteln:
$$P^2 = 1 \quad P = P^{-1} $$
$$P^2\ket{a} = \ket{a} = \pi_a^2\ket{a}$$
$$\rightarrow \pi_a = \pm 1$$

Der Erwartungswert eines beliebigen Operators mit definierter Parität transformiert sich unter der Paritätsoperation also wie
$$POP^{-1} = \pi_0 0$$
das heißt das Vorzeichen bleibt entweder erhalten oder ändert sich. 
Operatoren deren Erwartungswert das Vorzeichen ändert, sind unter dieser Transformation nicht invariant. Sie werden durch polare Vektoren beschrieben. Beispiele hierfür sind der Orts- und Impulsoperator.\\
Operatoren deren Erwartungswert das Vorzeichen nicht ändert, sind unter dieser Transformation invariant. Sie werden axiale Vektoren beschrieben. Beispiele hierfür sind der Drehimpuls- und Spinoperator. 
Die Paritätserhaltung beschreibt die Invarianz des Erwartungswerts eines Operators gegen Raumspiegelungen. Um die Paritätserhaltung zu prüfen, werden Operatoren gemessen, die empfindlich auf diese Raumspiegelung sind. Diese Operatoren nennt man Pseudoskalare. Sie entstehen aus dem Skalarprodukt eines axialen und eines polaren Vektors. Ist die Parität erhalten, sind diese Pseudoskalare immer notwendigerweise gleich Null. 
$$\braket{a} = \braket{PaP} = \braket{-a} \rightarrow a = 0$$
Ein von Null verschiedener Pseudoskalar bedeutet somit die Verletzung der Parität.
In diesem Versuch wird der Helizitätsoperator $H$ als Pseudoskalar untersucht. 
$$ H = \frac{\vec{\sigma}\cdot\vec{p}}{\left\vert\vec{\sigma}\cdot\vec{p}\right\vert}$$


\subsection{Polarisation}

\section{Experimenteller Aufbau und Messprinzip}
\subsection{Unterabschnitt}
\section{Durchführung}
\subsection{Unterabschnitt}