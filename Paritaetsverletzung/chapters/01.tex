\chapter{Auswertung, Fehlerrechnung und Diskussion der Messergebnisse}
\section{Asymmetrie der Summen}
Zunächst wird die Asymmetrie $E$ mithilfe der Formel \ref{Asymmetrie} mit den Summen berechnet. Hierfür werden zuerst jeweils alle zu $N_{-}$ und $N_{+}$ gehörenden Messwerte aufsummiert
 \begin{equation*}
     \overline{N}_{-} = \sum N_{-} = 428262 \qquad \overline{N}_{+} = \sum N_{+} = 416099 ~.
 \end{equation*}
 Als statistische Unsicherheit dieser Werte werden deren Wurzeln angenommen
  \begin{equation*}
     \sigma_{N_{-}} = 654.4173 \qquad \sigma_{N_{+}} = 645.0574 ~.
 \end{equation*}
 Mittels Gaußscher Fehlerfortpflanzung kann nun der statistische Fehler berechnet werden. Es ergibt sich
 \begin{equation*}
     \sigma_\text{stat} = \sqrt{\left(\frac{2N_{-}}{(N_{-}+N_{+})^{2}} \cdot \sigma_{N_{+}}\right)^{2} + \left(\frac{2N_{+}}{(N_{-}+N_{+})^{2}} \cdot \sigma_{N_{-}}\right)^{2}} = 0.0011 ~.
 \end{equation*}
Als systematische Unsicherheit wird ein Wert von $\sigma=120$ für den Zähler angenommen. Das entspricht etwas mehr als der mittleren Unsicherheit, die der Detektor selbst liefert.  Dadurch folgt für die systematische Unsicherheit der Asymmetrie:  
 \begin{equation*}
     \sigma_\text{sys} = \sqrt{\left(\frac{2N_{-}}{(N_{-}+N_{+})^{2}} \cdot \sigma \right)^{2} + \left(\frac{2N_{+}}{(N_{-}+N_{+})^{2}} \cdot \sigma \right)^{2}} = 0.0002 ~.
 \end{equation*}
Somit ergibt sich für die Asymmetrie $$E=0.0144 \pm 0.0011 \pm 0.0002$$
    

\section{Asymmerie der Einzelmessungen}
Mit dieser Methode soll die die Asymmetrie ermittelt werden, indem $E$ aus den einzelnen Messpaaren $N_{-}$ und $N_{+}$ berechnet und anschließend der Mittelwert gebildet wird. Die statistische Unsicherheit ist durch die Standardabweichung der Asymmetrien geteilt durch die Wurzel der Anzahl der Messungen gegeben $$\sigma_\text{stat}=0.0012~.$$ Der systematische Fehler wird mit 
 \begin{equation*}
     \sigma_\text{sys} = \frac{\sqrt{2(1+E^{2})}}{N_{-}+N_{+}} \cdot \sigma = 0.001
 \end{equation*}
mit $\sigma=120$ berechnet. Mit dieser Methode ergibt sich $$E=0.0144 \pm 0.001 \pm 0.0002$$
Sowohl die ermittelten Werte für $E$, die aus den beiden Methoden resultieren, als auch deren berechneten Unsicherheiten stimmen gut überein. 

\begin{table}[!htb]
    \centering
    \caption{Anzahl der Gammaquanten am Detektor}
    \label{}
    \begin{tabular}{c c c c c c}\\
\toprule
Messung & $N_{-}$ & $N_{+}$ & Messung & $N_{-}$  & $N_{+}$ \\
\hline
1 & 13397 & 13141 & 16 & 14129 & 13783\\
2 & 13690 & 13138 & 17 & 14075 & 14131\\
3 & 13816 & 13519 & 18 & 14570 & 13969\\
4 & 13856 & 13416 & 19 & 14394 & 13910\\
5 & 13748 & 13575 & 20 & 14398 & 13891\\
6 & 13694 & 13374 & 21 & 14604 & 14091\\
7 & 14119 & 13407 & 22 & 14598 & 14081\\
8 & 13989 & 13404 & 23 & 14799 & 14342\\
9 & 13993 & 13612 & 24 & 14806 & 14200\\
10 & 14045 & 13738 & 25 & 14828 & 14162\\
11 & 13756 & 13550 & 26 & 14926 & 14328\\
12 & 13966 & 13658 & 27 & 14924 & 14436\\
13 & 14059 & 13751 & 28 & 14806 & 14629\\
14 & 14135 & 13709 & 29 & 14870 & 14452\\ 
15 & 14135 & 13950 & 30 & 15137 & 14752\\
\bottomrule
    \end{tabular}
\end{table}



\section{Helizität}
Mit der aus den vorherigen Teilen ermittelten Asymmetrien kann nun die Polarisation $P_{G}$ der Gammaquanten bestimmt werden. Dies erfolgt mit Formel \ref{Polarisation}. Wie schon in der Vorbereitung beschrieben gilt $$\frac{\Phi_0}{\Phi_C^-}= 0.52$$ mit einer Unsicherheit $\sigma_{\frac{\Phi_0}{\Phi_C^-}}=0.05$ und $f=\frac{2}{26}$, wobei für den Polarisationsgrad $f$ eine Unsicherheit von $\sigma_{f}=0.007$ verwendet wird.\\
Die statistischen und systematischen Unsicherheiten auf die Polarisation können mit gaußscher Fehlerfortpflanzung ermittelt werden.
 $$\sigma_{P_{G,\text{stat}}}=\frac{\sigma_{E_\text{stat}}}{f \cdot \frac{\Phi_0}{\Phi_C^-}}~.$$ 
$$\sigma_{P_\text{sys}}=\sqrt{\left(\frac{1}{f \cdot \frac{\Phi_0}{\Phi_C^-}}\cdot \sigma_{E_\text{sys}} \right)^{2} + \left( -\frac{E}{f \cdot \frac{\Phi_0}{\Phi_C^-}^{2}}\cdot \sigma_{\frac{\Phi_0}{\Phi_C^-}}\right)^{2} + \left(\frac{E}{f^{2} \cdot \frac{\Phi_0}{\Phi_C^-}} \cdot \sigma_{f}\right)^{2}} ~.$$
Es ergeben sich die Polarisationen $P_{G,1}$ für die erste Methode und $P_{G,2}$ für die zweite Methode \begin{align*}
    P_{G,1}&=0.3601 \pm 0.0272 \pm 0.0479 \\
    P_{G,1}&=0.3593 \pm 0.0272 \pm 0.0478 ~.
\end{align*} 
Gemäß Formel \ref{Helizität} kann nun die Helizität $H$ berechnet werden. Der Wert für $L$, dem Helizitätsübertrag wird mit \\$L=0.8$, $\sigma_{L}=0.15$ abgeschätzt. Die Fehler werden mit Gaußscher Fehlerfortpflanzung berechnet. 
\begin{align*} 
    \sigma_{H,\text{stat}}&=\sqrt{\left(\frac{\sigma_{P_{G,\text{stat}}}}{L}\right)^{2}} \\
    \sigma_{H,\text{sys}}&=\sqrt{\left(\frac{1}{L}\cdot \sigma_{P_{G,\text{sys}}}\right)^{2} + \left(-\frac{P}{L^{2}}\cdot \sigma_{L}\right)^{2}}
\end{align*} 
Die Helizität beträgt 
\begin{align*} 
    H_{1}&=0.4502 \pm 0.0340 \pm 0.0479 \\
    H_{2}&=0.4492 \pm 0.0340 \pm 0.0478 ~.
\end{align*}
Es ist zu sehen, dass die Helizität auch unter Einbezug der Unsicherheiten ungleich Null ist und daher die Parität verletzt ist. Somit ist nachgewiesen, dass die Parität beim $\beta$-Zerfall der schwachen Kraft keine Erhaltungsgröße ist.