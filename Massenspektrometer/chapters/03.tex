\chapter{Durchführung}
\section{Einführung}
Der erste Versuchsteil dient dazu, sich mit dem Umgang mit dem Massenspektrometer und dem Vakuumsystem vertraut zu machen. Dazu wird ein Restgas-Spektrum im Massenbereich $1\,$amu - $200\,$amu unter Standardbedingungen, also $E=65\,$EV und $i=1\,$mA, aufgenommen. Es werden Peaks bei $m=18, 28, 32, 44$ sichtbar sein. Aufgrund dieser dominierenden Linien kann auf die Zusammensetzung des Restgases geschlossen werden. Es besteht eine gewisse Beziehung zwischen Partialdruck $p$ und Elektronenstrom $i$, daher wird als nächstes die Abhängigkeit des Partialdrucks vom Elektronenstrom für eine Linie untersucht. Zudem wird das Auflösungsvermögen  bestimmt. Hierfür wird die Linienbreite einer Linie bei kleiner und bei großer Massenzahl gemessen. 

\section{Auftrittsenergie von Argon}
In diesem Versuch soll die Auftrittsenergie der Argonionen $A^{+}$ und $A^{++}$ gemessen werden. Dafür wird die Kammer mit Argon gefüllt bis der Maximaldruck $p=5 \cdot 10^{-6}\,$mbar beträgt. Dann wird mithilfe des Oszilloskops der Partialdruck bei variierender Energie bis $100\,$eV aufgenommen.

\section{Dissoziationsenergien von Stickstoff}
Ähnlich wie im zweiten Versuchsteil werden nun die Dissoziationsenergien von Stickstoff bestimmt. Die Kammer wird diesmal mit Raumluft wieder bis zum Maximaldruck gefüllt. Für $m=28$ und $m=14$ wird analog zum vorherigen Versuch die Elektronenenergie variiert und dabei der Partialdruck gemessen. Anschließend können die Auftrittsenergien bestimmt werden, mit denen dann die Dissoziationsenergien berechnet werden können.

\section{Quantitative Analyse}
Nun wird die Zusammensetzung der Luft untersucht. Dazu wird die Kammer bis zum Maximaldruck belüftet und ein Spektrum aufgenommen, um die einzelnen Luftkomponenten zu analysieren. Zur Ermittlung der Zusammensetzung wird von jeder Gruppe die maximale Linie betrachtet und der Partialdruck bestimmt. Um das Ergebnis zu überprüfen, werden die Partialdrücke aufsummiert und mit dem Druck der Ionisationsmessröhre, der auf dem Ionisationsmanometer angezeigten Druck, verglichen. 

\section{Qualitative Analyse}
Um ein unbekanntes Gas $C_\text{x}H_\text{y}$ qualitativ zu analysieren, wird von diesem verschiedene Spektren bei Energien $E=15, 30, 65\,$eV aufgenommen. Mithilfe von Vergleichstabellen kann dann die Art des Gases bestimmt werden.

\section{Thermische Zersetzung}
Im letzten Versuchsteil soll die thermische Zersetzung des Feststoffs CaCO$_{3}$ analysiert werden. Durch Erhitzen von CaCO$_{3}$ auf $T=500$° zersetzt es sich in seine Bestandteile. Während des Erhitzens werden Spektren aufgenommen, mit denen, nach Vergleich dieser Spektren, Aussagen über die Zersetzung von Caliumcarbonat getroffen werden können. Beim Abkühlen des Feststoffs  wird der Partialdruck des gasförmigen Komponente in Abbhängigkeit der Temperatur gemessen. Daraus kann dann die Zersetzungsenthalpie $\Delta H$ bestimmt werden.