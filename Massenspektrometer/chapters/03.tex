\chapter{Durchführung}
\section{Einführung}
Der erste Versuchsteil dient dazu, sich mit dem Umgang mit dem Massenspektrometer und dem Vakuumsystem vertraut zu machen. Dazu wird ein Restgas-Spektrum im Massenbereich $1\,$amu - $200\,$amu unter Standardbedingungen, also $E=65\,$EV und $i=1\,$mA, aufgenommen. Es werden Peaks bei $m=18, 28, 32, 44$ sichtbar sein. Aufgrund dieser dominierenden Linien kann auf die Zusammensetzung des Restgases geschlossen werden. Es besteht eine gewisse Beziehung zwischen Partialdruck $p$ und Elektronenstrom $i$, daher wird als nächstes die Abhängigkeit des Partialdrucks vom Elektronenstrom für eine Linie untersucht. Zudem wird das Auflösungsvermögen  bestimmt. Hierfür wird die Linienbreite einer Linie bei kleiner und bei großer Massenzahl gemessen. 

\section{Auftrittsenergie von Argon}
In diesem Versuch soll die Auftrittsenergie der Argonionen $A^{+}$ und $A^{++}$ gemessen werden. Dafür wird die Kammer mit Argon gefüllt bis der Maximaldruck $p=5 \cdot 10^{-6}\,$mbar beträgt. Dann wird mithilfe des Oszilloskops der Partialdruck bei variierender Energie bis $100\,$eV aufgenommen.

\section{Dissoziationsenergien von Stickstoff}
Ähnlich wie im zweiten Versuchsteil werden nun die Dissoziationsenergien von Stickstoff bestimmt. Die Kammer wird diesmal mit Raumluft wieder bis zum Maximaldruck gefüllt. Für $m=28$ und $m=14$ wird analog zum vorherigen Versuch die Elektronenenergie variiert und dabei der Partialdruck gemessen. Anschließend können die Auftrittsenergien bestimmt werden, mit denen dann die Dissoziationsenergien berechnet werden können.

\section{Quantitative Analyse}
