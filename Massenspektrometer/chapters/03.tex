\chapter{Durchführung}
\section{Einführung}
Der erste Versuchsteil dient dazu, sich mit dem Umgang mit dem Massenspektrometer und dem Vakuumsystem vertraut zu machen. Dazu wird ein Restgas-Spektrum im Massenbereich $1\,$u $-2\,$u unter Standardbedingungen, also $E=65\,$EV und $I_\text{e}=1\,$mA, aufgenommen. Es werden Peaks bei $m=18, 28, 32, 44$ sichtbar sein. Aufgrund dieser dominierenden Linien kann auf die Zusammensetzung des Restgases geschlossen werden. 